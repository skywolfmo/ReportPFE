% @Author: Taha Bouhsine


\chapter*{Dedication}
\addcontentsline{toc}{chapter}{Dedication}
% \thispagestyle{empty}
%
%For all they have endured to satisfy all my needs and wishes
\vspace*{\fill}

\begin{center}
  In memoriam Hanya Laaiba, my dear grandmother, ~ \\
  her last words of encouragement are living with me to this day, ~ \\
  feeding my motivation, and always urging me to pursue my dreams, and my studies, ~\\
  never to be forgotten, never to be erased, ~ \\
  may your soul rest in peace, ~ \\
  may your memory guide me through my future steps to the unknown ~ \\


\end{center}
%
\nopagebreak{%
  % And maybe a quote here
  \raggedright\hspace{5.75cm} To your beautiful soul,~\\
  \raggedright\hspace{7.75cm} I dedicate this work\@.~\\~\\~\\
  %
  \raggedleft\normalfont\large\itshape{} \reportAuthor\par%
}
\vspace*{\fill}

%
%
%
%
%
%
\cleardoublepage%
\chapter*{Acknowledgements}
\addcontentsline{toc}{chapter}{Acknowledgements}
% \thispagestyle{empty}
%

I wish to express my deepest appreciation to my supervisor, \mentor, she convincingly guided and encouraged me to take the exact decisions and do the right things even when the road got tough. Without her persistent help, and profound belief in my abilities the goal of this project would not have been realized.

I wish to acknowledge the help provided by the administration staff at Ibn Zohr University, Faculty of Science, and would like to thank them for giving us a great studying environment. And to all my classmates and the professors of Computer Science Departement for having contributed to the formulation of our ideas and for providing a suitable working environment towards the completion of this project.

Finally, I must express my very profound gratitude to my parents for raising me, and to my friends for providing me with unfailing support and continuous encouragement throughout my years of study and throughout the process of researching and developing this project.
This accomplishment would not have been possible without them. Thank you.

\cleardoublepage%
\chapter*{Abstract}
\addcontentsline{toc}{chapter}{Abstract}
% \thispagestyle{empty}
%q

Crowdfunding refers to behavior where public individuals, rather than institutions, to use digital
technologies to make financial contributions to people, projects, or businesses in response to
either financial or developmental commitments from those people, projects, or businesses.
« Sahem » Crowdfunding Platform is a project that aims to develop a system that will be a gateway to allow project funders to
contribute towards a good project idea of a creator who lacks sufficient resources to implement
it. The main important role of this project is of it creating a link and  bridge the gap between
viable project ideas and successful implementation of
projects, so good project ideas can live longer, and benefit individuals and the community 
at large, and will not go into a waste.
With the aim to successfully develop this project, we will employ the use of the Software
development methodology, Iterations and Increments process, as well as we will be using the Kanban board Model
to divide and organize our project into small tasks, and also will be carrying out reviews and analysis of existing solutions in an attempt to
create a unique user experience with the use of Stripe API as a Payment Service alongside MongoDB, Express Js, Angular 9, and Node Js.


\cleardoublepage%
\chapter*{Résumé}
\addcontentsline{toc}{chapter}{Résumé}
% \thispagestyle{empty}
%
Le crowdfunding, appelé également financement participatif, est une technique de financement de projets utilisant internet comme canal de mise en relation entre les porteurs de projet et les personnes souhaitant investir dans ces projets.
La plateforme « Sahem » que nous proposons à travers ce projet, est une plateforme de financement participatif qui constitue une passerelle dont l’objectif est de permettre aux financeurs des projets de contribuer au financement d'une bonne idée de projet d’un créateur qui manque de ressources pour la mettre en œuvre. Ainsi, son rôle principal est de créer un lien pour combler cette fosse entre les idées de projets innovantes et la réussite de leur mise en œuvre. Pour le développement de ce projet, nous nous sommes basé sur une méthodologie de développement logiciel itérative et incrémentale, ainsi que sur le tableau Kanban pour diviser et organiser le projet en petites tâches. Pour la conception nous avons utilisé le langage UML et pour le développement nous avons utilisé plusieurs frameworks et outils tel que : Express Js, Angular 9, Node Js et MongoDB. Pour le service de paiement en ligne nous avons utilisé Stripe API.





